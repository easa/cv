%% start of file `template.tex'.
%% Copyright 2006-2015 Xavier Danaux (xdanaux@gmail.com).
%
% This work may be distributed and/or modified under the
% conditions of the LaTeX Project Public License version 1.3c,
% available at http://www.latex-project.org/lppl/.


\documentclass[12pt,a4paper,sans]{moderncv}


% moderncv themes
\moderncvstyle{banking}
\moderncvcolor{black}

% character encoding
\usepackage[utf8]{inputenc}                     % set encoding
\usepackage[scale=0.75]{geometry}               % customize paper
\usepackage[unicode]{hyperref}                  % to use hyperlinks
\usepackage{xcolor}                             % better colors
\usepackage{color}                              % syntactic sugar

% Setup bib formatting
\makeatletter\renewcommand*{\bibliographyitemlabel}{\@biblabel{\arabic{enumiv}}}\makeatother

% Setup fonts
\renewcommand*{\namefont}{\fontsize{24}{29}\mdseries\upshape}
\renewcommand{\familydefault}{\sfdefault}

% Setup bullet points
\renewcommand{\labelitemi}{\scriptsize\color{black}{$\bullet$}}


% PERSONAL DATA --------------------------------------------------------
\firstname{Lester James V.}
\familyname{Miranda}
\address{Kitakyushu-shi, Wakamatsu-ku, 1-15-4309 Hibikino, Fukuoka-ken, Japan}
\email{ljvmiranda@gmail.com}
\homepage{ljvmiranda921.github.io}
\social[github]{ljvmiranda921}
\extrainfo{Last update: \today}

\begin{document}

\maketitle


% EDUCATION ------------------------------------------------------------
\section{Education}
\cventry{2016--present}
        {M.Eng., Major in Information Architecture}
        {Waseda University}
        {Fukuoka, Japan}{}
        {}

\cventry{2011--2016}{B.S., Electronics \& Communications Engineering}
        {Ateneo de Manila University}
        {Quezon City, Philippines}
        {\textit{Cum Laude}}
        {Minor in Philosophy}

\cventry{Sep 2015 -- Jan 2016}
        {Exchange Student, Fall Semester}
        {Institut Catholique d'Arts et M\'etiers}
        {Lille, France}{}
        {}

% MASTER THESIS --------------------------------------------------------
\section{Master Thesis}
% The spacing here is better than using \cventry
\textbf{Title}: \emph{Autoencoder-based Feature Extraction Techniques for Protein Function Prediction}\\
\textbf{Supervisor}: Dr. Jinglu Hu (Furuzuki Takayuki)\\
\textbf{Laboratory}: {\color{blue}\httplink[Furuzuki Neurocomputing Systems Laboratory]{www.waseda.jp/sem-hflab/nclab/index.html}}\\
\textbf{Description}: We designed a mutually-competitive autoencoder architecture that learns
task-relevant representations of protein data to improve the prediction of protein functions.
We tested on \textit{S. cerevisiae} (yeast) and Genbase protein benchmarks and outperformed other
techniques in literature.

% EXPERIENCE -----------------------------------------------------------
\section{Work Experience}

\cventry{Jan 2018}
        {Research Intern}
        {RIKEN}
        {Kobe, Japan}{Center for Computational Sciences}
        {
        Implemented data assimilation techniques such as 3Dvar,
        4Dvar and Ensemble Kalman Filters (EnKF) to the Lorenz63 model
        for weather simulation. Participated in the RIKEN International
        School on Data Assimilation with support grant (one-week).
        }

\cventry{Apr 2015 -- May 2015}
        {Intern}
        {Manila Electric Company (MERALCO)}
        {Pasig, Philippines}{Strategy, Architecture \& Governance}
        {
        Developed business use-cases for 3rd IT Computing Platform
        Technologies (big data analytics, enterprise mobility, and drone
        automation) and presented it to MERALCO's top executives.
        }

\newpage
% PROJECTS -------------------------------------------------------------
\section{Selected Projects}

\cventry{2018}
        {\color{blue} \httplink[ljvmiranda921/gym-lattice]{github.com/ljvmiranda921/gym-lattice}}
        {Gym-Lattice}{}{}
        {
        A Python library that provides a reinforcement learning environment
        to solve the protein folding problem. Creates an OpenAI gym-like
        interface of the HP-Lattice structure in statistical mechanics.
        }

\cventry{2017}
        {\color{blue} \httplink[ljvmiranda921/pyswarms]{github.com/ljvmiranda921/pyswarms}}
        {PySwarms}{}{}
        {
        A Python-based framework for implementing swarm optimization
        algorithms. Features include visualization and hyperparameter
        optimization. Software paper was published in the Journal of Open
        Source Software (JOSS).
        }

\cventry{2017}
        {\color{blue} \httplink[scikit-multilearn]{github.com/scikit-multilearn/scikit-multilearn}}
        {Scikit-Multilearn}{}{}
        {
        A Python library that provides a \texttt{scikit-learn} interface for
        multilabel classification. Role: Collaborator. Responsibilities include
        resolving issues and updating library documentation.
        }

% SCHOLARSHIP ----------------------------------------------------------
\section{Scholarships Received}
\cvlanguage{}{Monbugakusho (MEXT) Japanese Government Scholarship}{2016}
\cvlanguage{}{French Ministry of Foreign and European Affairs Grant}{2015}
\cvlanguage{}{Department of Science \& Technology SEI Merit Scholarship}{2011}
\cvlanguage{}{Ateneo College Scholarship, 100 Tuition and Fees}{2011}
% BIBLIOGRAPHY ---------------------------------------------------------
\nocite{*}
\bibliographystyle{ieeetr}
\bibliography{publications}

\end{document}
