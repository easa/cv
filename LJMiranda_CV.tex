%% start of file `template.tex'.
%% Copyright 2006-2015 Xavier Danaux (xdanaux@gmail.com).
%
% This work may be distributed and/or modified under the
% conditions of the LaTeX Project Public License version 1.3c,
% available at http://www.latex-project.org/lppl/.


\documentclass[12pt,a4paper]{moderncv}


% moderncv themes
\moderncvstyle{banking}
\moderncvcolor{black}

% character encoding
\usepackage[utf8]{inputenc}                     % set encoding
\usepackage[scale=0.75]{geometry}               % customize paper
\usepackage[unicode]{hyperref}                  % to use hyperlinks
\usepackage{xcolor}                             % better colors
\usepackage{color}                              % syntactic sugar
% Font settings
\usepackage[T1]{fontenc}
% Serif body font
\usepackage{charter}
% Math font
\usepackage[charter]{mathdesign}
% Monospaced font
\usepackage{sourcecodepro}
% Sans-serif font
%\usepackage[lf]{FiraSans}

% Setup bib formatting
\makeatletter\renewcommand*{\bibliographyitemlabel}{\@biblabel{\arabic{enumiv}}}\makeatother

% Setup fonts
\renewcommand*{\namefont}{\fontsize{24}{29}\mdseries\upshape}
%\renewcommand{\familydefault}{\sfdefault}

% Setup bullet points
\renewcommand{\labelitemi}{\scriptsize\color{black}{$\bullet$}}


% PERSONAL DATA --------------------------------------------------------
\firstname{Lester James V.}
\familyname{Miranda}
\address{Quezon City, Metro Manila, Philippines}
\email{ljvmiranda@gmail.com}
%\mobile{(+63)905-258-1624}
\homepage{ljvmiranda921.github.io}
\social[github]{ljvmiranda921}
\extrainfo{Last updated: March 9, 2020}
\title{CV}

\begin{document}

\maketitle

% EXPERIENCE -----------------------------------------------------------
\section{Experience}
\cventry{Oct 2018 -- Present}
{Machine Learning Researcher}
{Thinking Machines Data Science}
{Philippines}{Machine Learning Team}
{
    \vspace{3px}
    \textit{Management}
    \begin{itemize}
        \item As Team Lead for the Document Processing Team, streamlined
            research and engineering efforts in the computer-vision and data processing
            space to enable fast delivery and execution of our
            {\color{blue}\httplink[DocumentAI]{https://thinkingmachin.es/services/doc-ai/}}
            product. (Apr 2020 -- present)
    \end{itemize}
    \vspace{3px}
    \textit{Client Projects}
    \begin{itemize}
        \item Developed multiple NLP-based products for a major investment firm
            in Singapore including: an internal search engine using BERT, 
            an OCR-based document processing tool for PDFs, and a Bloomberg industry
            classifier. (Feb 2019 -- Jul 2021)
        \item As Tech Lead, mentored a project team to deliver a large-scale
            digitization project of all LGUs across the country for the World Bank.
            (Jul 2020 -- Sep 2020)
        \item Developed a computer vision-based household detection model for
            one of the largest telecommunications player in the Philippines to
            aid their capex investments. (Oct 2018 -- Jun 2019)
    \end{itemize}
    \vspace{3px}
    \textit{Internal Efforts}
    \begin{itemize}
        \item Led the company's open-source initiatives through
            company-wide strategy, internal training sessions, co-development
            of open-source projects, and public talks. More information can be
            found at {\color{blue}
            \httplink[thinkingmachin.es/open-tm]{thinkingmachin.es/open-tm}}
        \item Led the development of an internal, cloud-based, machine learning
            platform using Argo and Kubernetes to speed-up research workflow,
            experimentation, and deployment.
    \end{itemize}
}

% INTERNSHIPS -----------------------------------------------------------
\subsection{Internships}

\cventry{Aug 2018 -- Sep 2018}
{Research Intern}
{Preferred Networks, Inc.}
{Japan}{ChainerRL Team}
{
    \begin{itemize}
        \item Developed a reinforcement learning parallelization
              framework based on batch Proximal Policy Optimization (PPO)
              for the open-source ChainerRL library.
    \end{itemize}
}

\cventry{Apr 2015 -- May 2015}
{Intern}
{Manila Electric Company}
{Philippines}{Strategy, Architecture \& Governance Team}
{
    \begin{itemize}
        \item Investigated use-cases for IT Computing platforms: big
              data, mobility, \& drone automation.
    \end{itemize}
}


% EDUCATION ------------------------------------------------------------
\section{Education}
\cventry{Sep 2016 -- Sep 2018}
{M.Eng., Major in Information Architecture}
{Waseda University}
{Japan}{}
{Thesis: Autoencoder-based Feature Extraction Techniques for Protein
    Function Prediction}

\cventry{Jun 2011 -- Jun 2016}{B.S., Electronics \& Communications Engineering}
{Ateneo de Manila University}
{Philippines}
{\textit{Cum Laude}}
{
    Thesis: Appliance Recognition using Hall-Effect Current Sensors for
    Power Management Systems\\
    Minor in Philosophy
}


\subsection{Fellowships \& Others}

\cventry{Jan 2018}
{Participant}
{RIKEN-Advanced Institute for Computational Sciences}
{Japan}{RIKEN International School for Data Assimilation}
{Studied data assimilation techniques (3DVar, Kalman Filters, etc.) for real-time numerical simulations.}

\cventry{Sep 2015 -- Jan 2016}
{Exchange Student, Fall Semester}
{Institut Catholique d'Arts et M\'etiers}
{France}{}
{Took courses in control systems and software development}

% PROJECTS -------------------------------------------------------------
\section{Selected Projects}

\subsection{Open-Source Software}


\cventry{2019}
{\color{blue}
    \httplink[thinkingmachines/geomancer]{github.com/thinkingmachines/geomancer}}
{Geomancer}{}{}
{
    A Python library to automate feature engineering of geospatial data. It
    harnesses a geospatial data source like OpenStreetMap (OSM) and a data
    warehouse like BigQuery.
}

\cventry{2018}
{\color{blue} \httplink[ljvmiranda921/gym-lattice]{github.com/ljvmiranda921/gym-lattice}}
{Gym-Lattice}{}{}
{
    A Python library that provides a reinforcement learning environment
    to solve the protein folding problem. Creates an OpenAI gym-like
    interface of the HP-Lattice structure in statistical mechanics.
}

\cventry{2017}
{\color{blue} \httplink[ljvmiranda921/pyswarms]{github.com/ljvmiranda921/pyswarms}}
{PySwarms}{}{}
{
    A Python-based framework for implementing swarm optimization
    algorithms. Software paper was published in the \textit{Journal of Open
        Source Software} (JOSS).
}


% AWARDS AND CERTIFICATIONS ----------------------------------------------------------
\section{Awards and Certifications}

\subsection{Professional Certifications}
\cvlanguage{}{Google Cloud Professional Data Engineer (Certification ID: enjfUz)}{2018}

\subsection{Scholarships}
\cvlanguage{}{Monbugakusho (MEXT) Japanese Government Scholarship}{2016}
\cvlanguage{}{French Ministry of Foreign and European Affairs Grant}{2015}
\cvlanguage{}{Department of Science \& Technology SEI Merit Scholarship}{2011}
\cvlanguage{}{Ateneo College Scholarship, 100\% Tuition and Fees (100TF)}{2011}


% BIBLIOGRAPHY ---------------------------------------------------------
\nocite{*}
\bibliographystyle{ieeetr}
\bibliography{publications}

\end{document}
